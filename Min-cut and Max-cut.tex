\documentclass[a4paper, 12pt, titlepage]{article}

\usepackage{amssymb}
\usepackage{amsmath}
\newtheorem{theorem}{Theorem}
\newtheorem{lemma}{Lemma}
\newtheorem{proof}{Proof}
\usepackage{mathtools}
\usepackage{indentfirst}
\setlength{\parindent}{2em}
\usepackage[CJKbookmarks]{hyperref}

\title{Notes for Min-Cut and Max-Cut}
\author{ritchie huang}
\date{2019-09-20}

\begin{document}

\maketitle 

\tableofcontents
\newpage

\section{Understand Karger's vanilla Min-Cut Algorithm}
Let \textbf{G} be the graph with \textbf{n} nodes, \textbf{C} be the min-cut. 

\begin{theorem}
\[
    Pr \bigg[ \text{\textbf{C} is returned} \bigg] \geq \frac{2}{n (n - 1)}
\]
\end{theorem}

\begin{proof}

In the ($i + 1$)~th contraction, $i = 0, 1, ..., n - 3$. There are only ({$n - i$}) nodes in the graph \textbf{G}, so the \textbf{edges space} to contract on is \textbf{at least} $\frac{(n - i)|\textbf{C}|}{2}$.
that is, if we denote the size of edges space as $ m $, we have $ m \geq \frac{(n - i) |\textbf{C}|}{2} $. 

Therefore, we obtain:

\begin{equation}
    \begin{aligned}
    Pr \bigg[ \text{sets of edges contracted in $(i + 1)$~th step} \cap \textbf{C} = \emptyset \bigg] &= 1 - \frac{|\textbf{C}|}{m} \\
                                                                                                      &\geq 1 - \frac{|\textbf{C}|}{\frac{(n - i) |\textbf{C}|}{2}} \\
                                                                                                      &= 1 - \frac{2}{n -i}
    \end{aligned}
\end{equation}

Combining all the ($n - 2$) steps together:

\begin{equation}\label{product prob}
    \begin{aligned}
    Pr \bigg[ return \text{\textbf{C}} \bigg] &= Pr \bigg[ \text{sets of edges contracted in (n - 2) steps} \cap \textbf{C} = \emptyset \bigg] \\
                                                    &\geq \prod_{i = 0}^{n - 3} (1 - \frac{2}{n - i}) \\
                                                    &= \frac{2}{n ( n - 1)}
    \end{aligned}
\end{equation}

\end{proof}

If we run the algorithm independently for $\frac{n (n - 1)}{2} \log n$ times.
\begin{equation}
    \begin{aligned}
        Pr \bigg[ \text{return \textbf{C at least once}} \bigg] &= 1 - Pr \bigg[ \text{all runs fail to return \textbf{C}} \bigg] \\
                                                                           &\geq 1 - \bigg( 1 - \frac{2}{n ( n - 1)} \bigg)^{\frac{n (n - 1)}{2} \log n} \\
                                                                           &= 1 - \bigg( \frac{1}{e} \bigg)^{log n} \\
                                                                           &= 1 - \frac{1}{n}
    \end{aligned}
\end{equation}
According to the proof, we are expected to get the correct min-cut with probability \textbf{more than} $1 - \frac{1}{n}$.

\section{Understand Fast Min-Cut Algorithm}

The key mind to get a faster algorithm is to Stop the vanilla algorithm at $k$-th step in a single run, where $k$ is to be specified.
How can we understand this \textbf{early stop} ? 

If we expand the equation \ref{product prob}:
\begin{equation}
    Pr \bigg[ return \text{\textbf{C}} \bigg] = (1 - \frac{2}{n})(1 - \frac{2}{n - 1})(1 - \frac{2}{n - 2}) \cdots (1- \frac{2}{4})(1 - \frac{2}{3})
\end{equation}

observe the right-hand side, we can find that with the increase of terms, the probability changes accordingly.

\end{document}