\documentclass[titlepage,a4paper,12pt]{article}
% packages
\usepackage{mathtools}
\usepackage{amssymb}
% 添加首行缩进,两个字符
\usepackage{indentfirst}
\setlength{\parindent}{2em}
\usepackage[CJKbookmarks]{hyperref}

% title
\title{Hessian Free Optimization}
\author{ritchie huang}
\date{2019-09-09}

\begin{document}

\maketitle 

\tableofcontents
\newpage

\section{Background}\label{sec: background}

we can get approximation of f(x) in second-order Tylor expansion:

\begin{equation}
	f(x+\Delta x) \approx f(x)+\nabla f(x)^{T} \Delta x+\Delta x^{T} H(f) \Delta x
\end{equation}

\section{Conjugate Gradient}\label{sec: conjugate gradient}

Suppose we define function as bellow:

\begin{equation}
	f(x) = \frac{1}{2} x^{T} A x + b^{T} x + c. \ x, b, c \in \mathbb{R}^n
\end{equation}

where $A^{T} = A$, that's say $A$ is symmetric.

\subsection{Prerequisite}

Taking gradient of $f$, we obtain:
\begin{equation}
	\nabla f(x) = A x + b
\end{equation}
the direction of gradient is the direction in which the function rises fatest.

If we hope to minimize our function, we can then update $x$ by:
\begin{equation}
	x = x - \alpha \nabla f(x)
\end{equation}
where $\alpha$ is called step-size (may be learning rate), we perform a line search to find the best $\alpha$ when the direction $d_0 = \nabla f(x)$ is given.

Note that choosing the best $\alpha$ is equivalent to minimize the following function:
\begin{equation}
\begin{aligned}
	g(\alpha) &= f(x_0 + \alpha d_0)\\
			  &= \frac{1}{2} (x_0 + \alpha d_0)^{T} A (x_0 + \alpha d_0) + b^{T} (x_0 + \alpha d_0) + c \\
			  &= \frac{1}{2} \alpha^2 d_0^{T} A d_0 + d_0^{T} (A x_0 + b) \alpha + \left( \frac{1}{2} x_0^{T} A x_0 + b^{T} x_0 + c \right)
\end{aligned}
\end{equation}

More generally, when we update $x_i$ iteratively, since $g(\alpha)$ is a quadratic function in $\alpha$, it has a unique global minimum or maximum.Assume this function has a global minimum where $g^{\prime}(\alpha) = 0$:
\begin{equation}
	g^{\prime}(\alpha) = \alpha (d_i^{T} A d_i) + d_i^{T} (A x_i + b) = 0
\end{equation}

Solving the equation above, we obtain:
\begin{equation}
	\alpha = - \frac{d_i^{T} (A x_i + b)}{d_i^{T} A d_i}
\end{equation}

Thus, we can update $x1$ using $x0$ and $\alpha$ in the iterative algorithm:
\begin{equation}
	x_{1} = x_0 - \alpha \nabla f(x_0)
\end{equation}

However, it's subtle that each $\alpha_i$ is related to $d_i = -\nabla f(x_i)$, when we are going to update $x_{i + 1}$, we may ruin our update from previous iteration.Therefore, we need to rectify direction $d_{i + 1}$, which is conjugate to $d_i$.
\begin{equation}
	d_{i + 1} = - \nabla f(x_{i + 1}) + \beta_i d_i
\end{equation}

But, what's $\beta_i$ ? We can derive it from the conjugacy between $d_{i + 1}$ and $d_i$.

We define vector $x$ and $y$ to be conjugate w.r.t a semi-definite matrix $A$ if $x^{T} A y = 0$.

We obtain that: 
\begin{equation}
\begin{aligned}
	d_{i + 1}^{T} A d_i &= 0 \\
						&= \left( - \nabla f(x_{i + 1}) + \beta_i d_i \right)^{T} A d_i \\
						&= -\nabla f(x_{i + 1})^{T} A d_i + \beta_i d_i^{T} A d_i
\end{aligned}
\end{equation}

Solve the equation above:
\begin{equation}
	\beta_i = \frac{\nabla f(x_{i + 1})^{T} A d_i}{d_i^{T} A d_i}
\end{equation}
\end{document}
